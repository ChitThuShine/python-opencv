% Base on http://wch.github.io/latexsheet/}{http://wch.github.io/latexsheet/
\documentclass[10pt,landscape, a4paper]{article}
\usepackage{multicol}
\usepackage{calc}
\usepackage{ifthen}
\usepackage[landscape]{geometry}
\usepackage{hyperref}
\usepackage{minted}
\usepackage{amsmath}
% rescale the whole thing
\usepackage[]{datetime2}


% This sets page margins to .5 inch if using letter paper, and to 1cm
% if using A4 paper. (This probably isn't strictly necessary.)
% If using another size paper, use default 1cm margins.
\ifthenelse{\lengthtest { \paperwidth = 11in}}
	{ \geometry{top=.5in,left=.5in,right=.5in,bottom=.5in} }
	{\ifthenelse{ \lengthtest{ \paperwidth = 297mm}}
		{\geometry{top=1cm,left=1cm,right=1cm,bottom=1cm} }
		{\geometry{top=1cm,left=1cm,right=1cm,bottom=1cm} }
	}

% Turn off header and footer
\pagestyle{empty}
 


% Redefine section commands to use less space
\makeatletter
\renewcommand{\section}{\@startsection{section}{1}{0mm}%
                                {-1ex plus -.5ex minus -.2ex}%
                                {0.5ex plus .2ex}%x
                                {\normalfont\large\bfseries}}
\renewcommand{\subsection}{\@startsection{subsection}{2}{0mm}%
                                {-1explus -.5ex minus -.2ex}%
                                {0.5ex plus .2ex}%
                                {\normalfont\normalsize\bfseries}}
\renewcommand{\subsubsection}{\@startsection{subsubsection}{3}{0mm}%
                                {-1ex plus -.5ex minus -.2ex}%
                                {1ex plus .2ex}%
                                {\normalfont\small\bfseries}}
\makeatother

% Define BibTeX command
\def\BibTeX{{\rm B\kern-.05em{\sc i\kern-.025em b}\kern-.08em
    T\kern-.1667em\lower.7ex\hbox{E}\kern-.125emX}}

% Don't print section numbers
\setcounter{secnumdepth}{0}


\setlength{\parindent}{0pt}
\setlength{\parskip}{0pt plus 0.5ex}


% -----------------------------------------------------------------------

\begin{document}

\raggedright
\footnotesize
\begin{multicols}{2}


% multicol parameters
% These lengths are set only within the two main columns
%\setlength{\columnseprule}{0.25pt}
\setlength{\premulticols}{1pt}
\setlength{\postmulticols}{1pt}
\setlength{\multicolsep}{1pt}
\setlength{\columnsep}{2pt}

\begin{center}
     \Large{\textbf{OpenCV 4.x Cheat Sheet (Python version)}} \\
     \small{A summary of: \url{https://docs.opencv.org/master/}}
\end{center}

\section{I/O}
\begin{tabular}{@{}ll@{}}
\mintinline{python}{i = imread("name.png")}    & Loads image as BGR (if grayscale, \texttt{B=G=R})\\
\mintinline{python}{i = imread("name.png", IMREAD_UNCHANGED)}    & Loads image as is (inc.\ transparency if available)\\
\mintinline{python}{i = imread("name.png", IMREAD_GRAYSCALE)} & Loads image as grayscale\\
\mintinline{python}{imshow("Title", i)} & Displays image $I$\\
\mintinline{python}{imwrite("name.png", i)} & Saves image $I$\\
\mintinline{python}{waitKey(500)} & Wait 0.5 seconds for keypress (0 waits forever)\\
\mintinline{python}{destroyAllWindows()} & Releases and closes all windows\\
\end{tabular}


\subsection{Color/Intensity}
\newlength{\MyLen}
%\settowidth{\MyLen}{\texttt{letterpaper}/\texttt{a4paper} \ }
%\begin{tabular}{@{}p{\the\MyLen}%
%                @{}p{\linewidth-\the\MyLen}@{}}
\begin{tabular}{@{}ll@{}}
\mintinline{python}{i_gray = cvtColor(i, COLOR_BGR2GRAY)}& BGR to gray conversion\\
\mintinline{python}{i_rgb = cvtColor(i, COLOR_BGR2RGB)}& BGR to RGB (useful for \mintinline{python}{matplotlib})\\
\mintinline{python}{i = cvtColor(i, COLOR_GRAY2RGB)}& Converts grayscale to RGB (\texttt{R=G=B})\\
\mintinline{python}{i = equalizeHist(i)}& Histogram equalization\\
\mintinline{python}{i = normalize(i, None, 0, 255, NORM_MINMAX, CV_8U)} & Normalizes $I$ between 0 and 255\\
\mintinline{python}{i = normalize(i, None, 0, 1, NORM_MINMAX, CV_32F)} & Normalizes $I$ between 0 and 1
\end{tabular}
\subsubsection{Other useful color spaces}
\begin{tabular}{@{}ll@{}}
\mintinline{python}{COLOR_BGR2HSV}& BGR to HSV (Hue, Saturation, Value)\\
\mintinline{python}{COLOR_BGR2LAB}& BGR to Lab (Lightness, Green/Magenta, Blue/Yellow)\\
\mintinline{python}{COLOR_BGR2LUV}& BGR to Luv ($\approx$ Lab, but different normalization)\\
\mintinline{python}{COLOR_BGR2YCrCb}& BGR to YCrCb (Luma, Blue-Luma, Red-Luma)\\
\end{tabular}

\subsection{Channel manipulation}
\begin{tabular}{@{}ll@{}}
\mintinline{python}{b, g, r = split(i)}& Splits the image $I$ into channels\\
\mintinline{python}{b, g, r, a = split(i)}& Same as above, but $I$ has alpha channel\\
\mintinline{python}{i = merge((b, g, r))}& Merges channels into image\\


\end{tabular}

\subsection{Arithmetic operations}
\begin{tabular}{@{}ll@{}}
\mintinline{python}{i = add(i1, i2)}& $\min(I_1 + I_2, 255)$, i.e.\ saturated addition if \texttt{uint8}\\
\mintinline{python}{i = addWeighted(i1, alpha, i2, beta, gamma)}& $\min(\alpha I_1 + \beta I_2 + \gamma, 255)$, i.e.\ image blending\\
\mintinline{python}{i = subtract(i1, i2)}& $\max(I_1 - I_2, 0)$, i.e.\ saturated subtraction if \texttt{uint8}\\
\mintinline{python}{i = absdiff(i1, i2)}& $\left| I_1 - I_2\right|$, i.e.\ absolute difference\\
\end{tabular}

\textbf{Note:} one of the images can be replaced by a scalar.


\subsection{Logical operations}
\begin{tabular}{@{}ll@{}}
    \mintinline{python}{i = bitwise_not(i)}& Inverts every bit in $I$ (e.g.\ mask inversion)\\
    \mintinline{python}{i = bitwise_and(i1, i2)}& Logical \textit{and} between $I_1$ and $I_2$ (e.g.\ mask image)\\
    \mintinline{python}{i = bitwise_or(i1, i2)}& Logical \textit{or} between $I_1$ and $I_2$ (e.g.\ merge 2 masks)\\
    \mintinline{python}{i = bitwise_xor(i1, i2)}& Exclusive \textit{or} between $I_1$ and $I_2$\\
\end{tabular}

\subsection{Statistics}
\begin{tabular}{@{}ll@{}}
	\mintinline{python}{mB, mG, mR, mA = mean(i)} & Average of each channel (i.e.\ BGRA)\\
	\mintinline{python}{ms, sds = meanStdDev(i)} & Mean and SDev p/channel (3 or 4 rows each)\\
	\mintinline{python}{h = calcHist([i], [c], None, [256], [0,256])} & Histogram of channel \texttt{c}, no mask, 256 bins (0-255)\\
	%\mintinline{python}{h = calcHist([i], [0,1], None, [256,256], [0,256, 0,256])} & 2D histogram using channels 0, 1\\
    \mintinline{python}{h = calcHist([i], [0,1], None, [256,256],} & 2D histogram using channels 0 and 1, with\\
    \multicolumn{1}{r}{\mintinline{python}{[0,256, 0,256])}}&\phantom{ } ``resolution'' 256 in each dimension\\
\end{tabular}

\subsection{Filtering}
\begin{tabular}{@{}ll@{}}
\mintinline{python}{i = blur(i, (5, 5))} & Filters $I$ with $5\times 5$ box filter (i.e.\ average filter)\\
\mintinline{python}{i = GaussianBlur(i, (5,5), sigmaX=0, sigmaY=0)} & Filters $I$ with $5\times 5$ Gaussian; auto $\sigma$s; ($I$ is \mintinline{python}{float})\\
\mintinline{python}{i = GaussianBlur(i, None, sigmaX=2, sigmaY=2)} & Blurs, auto kernel dimension\\
\mintinline{python}{i = filter2D(i, -1, k)} & Filters with 2D kernel using cross-correlation\\
\mintinline{python}{kx = getGaussianKernel(5, -1)} & 1D Gaussian kernel with length 5 (auto StDev)\\
\mintinline{python}{i = sepFilter2D(i, -1, kx, ky)} & Filter using separable kernel (same output type)\\
\mintinline{python}{i = medianBlur(i, 3)} & Median filter with size=3 (size $\geq 3$)\\
\mintinline{python}{i = bilateralFilter(i, -1, 10, 50)} & Bilateral filter with $\sigma_\text{r} = 10$, $\sigma_\text{s}=50$, auto size\\
\end{tabular}
\subsubsection{Borders}
All filtering operations have parameter \mintinline{python}{borderType} which can be set to:
\begin{tabular}{@{}ll@{}}
\mintinline{python}{BORDER_CONSTANT} & Pads with constant border (requires additional parameter \mintinline{python}{value})\\
\mintinline{python}{BORDER_REPLICATE} & Replicates the first/last row and column onto the padding\\
\mintinline{python}{BORDER_REFLECT} & Reflects the image borders onto the padding\\
\mintinline{python}{BORDER_REFLECT_101} & Same as previous, but doesn't include the pixel at the border (the default)\\
\mintinline{python}{BORDER_WRAP} & Wraps around the image borders to build the padding\\
\end{tabular}

Borders can also be added with custom widths:
\begin{tabular}{@{}ll@{}}
\mintinline{python}{i = copyMakeBorder(i, 2, 2, 3, 1, borderType=BORDER_WRAP)} & Widths: top, bottom, left, right\\\\
\end{tabular}

\subsection{Differential operators}
\begin{tabular}{@{}ll@{}}
\mintinline{python}{i_x = Sobel(i, CV_32F, 1, 0)} & Sobel in the x-direction: $I_x = \frac{\partial}{\partial x}I$\\
\mintinline{python}{i_y = Sobel(i, CV_32F, 0, 1)} & Sobel in the y-direction: $I_y = \frac{\partial}{\partial y}I$\\
\mintinline{python}{i_x, i_y = spatialGradient(i, 3)} & The gradient: $\nabla I$ (using $3\times 3$ Sobel): needs \mintinline{python}{uint8} image\\
\mintinline{python}{m = magnitude(i_x, i_y)} & $\lVert\nabla I\rVert$; $I_x, I_y$ must be float (for conversion, see \mintinline{python}{np.astype()})\\
\mintinline{python}{m, d = cartToPolar(i_x, i_y)} & $\lVert\nabla I\rVert$; $\theta \in [0, 2\pi]$; \mintinline{python}{angleInDegrees=False}; needs \mintinline{python}{float32} $I_x, I_y$\\
\mintinline{python}{l = Laplacian(i, CV_32F, ksize=5)} & $\Delta I$, Laplacian with kernel size of 5\\
\end{tabular}

\subsection{Geometric transforms}
\begin{tabular}{@{}ll@{}}
    \mintinline{python}{i = resize(i, (width, height))} & Resizes image to \texttt{width}$\times$\texttt{height}\\
    \mintinline{python}{i = resize(i, None, fx=0.2, fy=0.1)} & Scales image to 20\% width and 10\% height\\
    \mintinline{python}{M = getRotationMatrix2D((xc, yc), deg,} & Returns $2\times 3$ rotation matrix \texttt{M}, arbitrary $(x_c, y_c)$\\
    \multicolumn{1}{r}{\mintinline{python}{scale)}} &\\
    \mintinline{python}{M = getAffineTransform(pts1,pts2)} & Affine transform matrix \texttt{M} from 3 correspondences\\
    \mintinline{python}{i = warpAffine(i, M, (cols,rows))} & Applies Affine transform \texttt{M} to $I$, output size=(\texttt{cols}, \texttt{rows}) \\
    \mintinline{python}{M = getPerspectiveTransform(pts1,pts2)} & Perspective transform matrix \texttt{M} from 4 correspondences\\
    \mintinline{python}{M, s = findHomography(pts1, pts2)} & Persp transf mx \texttt{M} from all $\gg 4$ corresps (Least squares)\\
    \mintinline{python}{M, s = findHomography(pts1, pts2, RANSAC)} & Persp transf mx \texttt{M} from best $\gg 4$ corresps (RANSAC)\\
    \mintinline{python}{i = warpPerspective(i, M, (cols, rows))} & Applies perspective transform \texttt{M} to image $I$\\
\end{tabular}
\subsubsection{Interpolation methods}
\mintinline{python}{resize}, \mintinline{python}{warpAffine} and \mintinline{python}{warpPerspective} use bilinear interpolation by default. It can be changed by parameter \mintinline{python}{interpolation} for \mintinline{python}{resize}, and \mintinline{python}{flags} for the others:
\begin{tabular}{@{}ll@{}}
    \mintinline{python}{flags=INTER_NEAREST} & Simplest, fastest (or \mintinline{python}{interpolation=INTER_NEAREST})\\
    \mintinline{python}{flags=INTER_LINEAR} & Bilinear interpolation: Default\\
    \mintinline{python}{flags=INTER_CUBIC} & Bicubic interpolation\\
\end{tabular}


\subsection{Segmentation}
\begin{tabular}{@{}ll@{}}
\mintinline{python}{_, i_t = threshold(i, t, 255, THRESH_BINARY)} & Manually thresholds image $I$ given threshold level $t$\\
\mintinline{python}{t, i_t = threshold(i, 0, 255, THRESH_OTSU)} & Returns thresh level and thresholded image using Otsu\\
\mintinline{python}{i_t = adaptiveThreshold(i, 255, } & \\
    \multicolumn{1}{r}{\mintinline{python}{ADAPTIVE_THRESH_MEAN_C, THRESH_BINARY, b, c)}}& Adaptive mean-c with block size $b$ and constant $c$\\
\mintinline{python}{bp = calcBackProject([i_hsv], [0,1], h,} & Back-projects histogram $h$ onto the image \texttt{i\_hsv}\\
\multicolumn{1}{r}{\mintinline{python}{ [0,180, 0,256], 1)}}&\phantom{ }  using only hue and saturation; no scaling (i.e.\ 1)\\    
\mintinline{python}{cp, la, ct = kmeans(feats, K, None, crit, 10,} & Returns the labels \texttt{la} and centers \texttt{ct} of \texttt{K} clusters,\\
\multicolumn{1}{r}{\mintinline{python}{KMEANS_RANDOM_CENTERS)}}&\phantom{ } best compactness \texttt{cp} out of 10; 1 feat/column\\    
\end{tabular}


\subsection{Features}
\begin{tabular}{@{}ll@{}}
\mintinline{python}{e = Canny(i, tl, th)} & Returns the Canny edges (\texttt{e} is binary)\\
\mintinline{python}{l = HoughLines(e, 1, pi/180, 150)} & Returns all $(\rho, \theta) \geq 150$ votes, Bin res: $\rho = 1$ pix, $\theta = 1\deg$\\
\mintinline{python}{l = HoughLinesP(e, 1, pi/180, 150,}&\\
 \multicolumn{1}{r}{\mintinline{python}{None, 100, 20)}}   & Probabilistic Hough, min length=100, max gap=20\\
 \mintinline{python}{c = HoughCircles(i, HOUGH_GRADIENT, 1,} & Returns all $(x_c, y_c, r)$ with at least 18 votes, bin resolution=1,\\
 \multicolumn{1}{r}{\mintinline{python}{minDist=50, param1=200, param2=18,}}  & \phantom{ }  param1 is the $t_h$ of Canny, and the centers must be at least\\
  \multicolumn{1}{r}{\mintinline{python}{minRadius=20, maxRadius=60)}}   & \phantom{ } 50 pixels away from each other\\
\mintinline{python}{r = cornerHarris(i, 3, 5, 0.04)} & Harris corners' $R$s per pixel, window=3, Sobel=5, $\alpha=0.04$\\
\end{tabular}
\begin{tabular}{@{}ll@{}}
\mintinline{python}{f = FastFeatureDetector_create()} & Instantiates the Star feature detector\\
\mintinline{python}{k = f.detect(i, None)} & Detects keypoints on grayscale image $I$\\
\mintinline{python}{i_k = drawKeypoints(i, k, None)} & Draws keypoints \texttt{k} on color image $I$\\
\mintinline{python}{d = xfeatures2d.BriefDescriptorExtractor_create()} & Instantiates a BRIEF descriptor\\
\mintinline{python}{k, ds = d.compute(i, k)} & Computes the descriptors of keypoints \texttt{k} over $I$\\
\mintinline{python}{dd = AKAZE_create()} & Instantiates the AKAZE detector/descriptor\\
\mintinline{python}{m = BFMatcher.create(NORM_HAMMING,} & Instantiates a brute-force matcher,\\
\multicolumn{1}{r}{\mintinline{python}{crossCheck=True)}}   & \phantom{ }with x-checking, and Hamming distance\\
\mintinline{python}{ms = m.match(ds_l, ds_r)} & Matches the left and right descriptors\\
\mintinline{python}{i_m = drawMatches(i_l, k_l, i_r, k_r, ms, None)} & Draws matches from the left keypoints \texttt{k\_l} on\\
&\phantom{ } left image $I_l$ to right $I_r$, using matches \texttt{ms}\\
\end{tabular}

\subsection{Detection}
\begin{tabular}{@{}ll@{}}
\mintinline{python}{ccs = matchTemplate(i, t, TM_CCORR_NORMED)} & Matches template $T$ to image $I$ (normalized X-correl)\\
\mintinline{python}{m, M, m_l, M_l = minMaxLoc(ccs)} & Min, max values and respective coordinates in \texttt{ccs}\\
\mintinline{python}{c = CascadeClassifier()} & Creates an instance of an ``empty'' cascade classifier\\
\mintinline{python}{r = f.load("file.xml")}& Loads a pre-trained model from file; \texttt{r} is \mintinline{python}{True/False}\\
\mintinline{python}{objs = f.detectMultiScale(i)} & Returns 1 tuple \texttt{(x, y, w, h)} per detected object\\
\end{tabular}

\subsection{Motion and Tracking}
\begin{tabular}{@{}ll@{}}
\mintinline{python}{pts = goodFeaturesToTrack(i, 100, 0.5, 10)} & Returns 100 Shi-Tomasi corners with, at least, 0.5\\
&\phantom{ }quality, and 10 pixels away from each other\\
\mintinline{python}{pts1, st, e = calcOpticalFlowPyrLK(i0, i1,}& New positions of pts from estimated optical\\
\multicolumn{1}{r}{\mintinline{python}{pts0, None)}}&flow between $I_0$ and $I_1$; \texttt{st[i]} is 1 if flow\\    
&\phantom{ }for point \texttt{i} was found, or 0 otherwise\\
\mintinline{python}{t = TrackerCSRT_create()} & Instantiates the CSRT tracker\\
\mintinline{python}{r = t.init(f, bbox)} & Initializes tracker with frame and bounding box\\
\mintinline{python}{r, bbox = t.update(f)} & Returns new bounding box, given next frame\\
\end{tabular}





\subsection{Drawing on the image}
\begin{tabular}{@{}ll@{}}
	\mintinline{python}{line(i,(x0, y0),(x1, y1), (b, g, r), t)}& Line\\
	\mintinline{python}{rectangle(i, (x0, y0), (x1, y1), (b, g, r), t)}& Rectangle\\
	\mintinline{python}{circle(i,(x0, y0), radius, (b, g, r), t)}& Circle\\
	\mintinline{python}{polylines(i,[pts], True, (b, g, r), t)}& Closed (\mintinline{python}{True}) polygon (\mintinline{python}{pts} is array of points)\\
	\mintinline{python}{putText(i, "Hi", (x,y), FONT_HERSHEY_SIMPLEX,}\\
      \multicolumn{1}{r}{\mintinline{python}{1, (r,g,b), 2, LINE_AA)}}& Writes ``Hi'' at $(x, y)$, font size=1, thickness=2\\    

\end{tabular}
\subsubsection{Parameters}
\begin{tabular}{@{}ll@{}}
	\mintinline{python}{(x0, y0)} & Origin/Start/Top left corner (note that it's not (row,column))\\
	\mintinline{python}{(x1, y1)} & End/Bottom right corner\\
	\mintinline{python}{(b, g, r)} & Line color (\mintinline{python}{uint8})\\
	\mintinline{python}{t} & Line thickness (fills, if negative)
\end{tabular}


\subsection{Calibration and Stereo}
\begin{tabular}{@{}ll@{}}
    %\mintinline{python}{s = cv2.StereoSGBM_create(minDisparity = 0, numDisparities = 32, blockSize = 11)} & initializes \\
    %
    \mintinline{python}{r, crns = findChessboardCorners(i, (n_x,n_y))} & 2D coords of detected corners; \mintinline{python}{i} is gray; \mintinline{python}{r} is\\
    \multicolumn{1}{r}{}   & \phantom{ }the status; \mintinline{python}{(n_x, n_y)} is size of calib target\\
    \mintinline{python}{crnrs = cornerSubPix(i, crns, (5,5), (-1,-1), crit)} & Improves coordinates with sub-pixel accuracy\\
    \mintinline{python}{r, K, D, ExRs, ExTs = calibrateCamera(crns_3D,}& Calculates intrinsics (inc. distortion coeffs), \&\\
    \multicolumn{1}{r}{\mintinline{python}{crns_2D, i.shape[:2], None, None)}}   &\phantom{ }extrinsics (i.e.\ \texttt{1 R+T} per target view); \mintinline{python}{crns_3D}\\
    \multicolumn{1}{r}{}   & \phantom{ }contains 1 array of 3D corner coords p/target\\
    \multicolumn{1}{r}{}   & \phantom{ }view; \mintinline{python}{crns_2D} contains the respective arrays of\\
    \multicolumn{1}{r}{}   & \phantom{ }2D corner coordinates (i.e.\ 1 \mintinline{python}{crns} p/target view)\\
    \mintinline{python}{drawChessboardCorners(i, (n_x, n_y), crns, r)} & Draws corners on $I$ (may be color); \texttt{r} is status\\
    \multicolumn{1}{r}{}   & \phantom{ } from corner detection\\
    \mintinline{python}{u = undistort(i, K, D)} & Undistorts $I$ using the intrinsics\\
    
    \mintinline{python}{s = StereoSGBM_create(minDisparity = 0,}&\\
    \multicolumn{1}{r}{\mintinline{python}{numDisparities = 32, blockSize = 11)}}   & Instantiates Semi-Global Block Matching method\\
    \mintinline{python}{s = StereoBM_create(32, 11)} & Instantiates a simpler block matching method\\
    \mintinline{python}{d = s.compute(i_L, i_R)} & Computes disparity map ($\propto^{-1}$ depth map)\\   
\end{tabular}

\subsection{Termination criteria (used in e.g.\ K-Means, Camera calibration)}
\begin{tabular}{@{}ll@{}}
    \mintinline{python}{crit = (TERM_CRITERIA_MAX_ITER, 20, 0)}& Stops after 20 iterations\\
    \mintinline{python}{crit = (TERM_CRITERIA_EPS, 0, 1.0)}& Stop if ``movement'' is less than 1.0\\
    \mintinline{python}{crit = (TERM_CRITERIA_MAX_ITER | TERM_CRITERIA_EPS, 20, 1.0)}& Stops whatever happens first\\
\end{tabular}


\subsection{Useful stuff}
\subsubsection{Numpy (\mintinline{python}{np.})}
\begin{tabular}{@{}ll@{}}
\mintinline{python}{m = mean(i)} & Mean/average of array $I$\\
\mintinline{python}{m = average(i, weights)} & Weighted mean/average of array $I$\\
\mintinline{python}{v = var(i)} & Variance of array/image $I$\\
\mintinline{python}{s = std(i)} & Standard deviation of array/image $I$\\
\mintinline{python}{h,b = histogram(i.ravel(),256,[0,256])} & \texttt{numpy} histogram also returns the bins \texttt{b}\\
\mintinline{python}{i = clip(i, 0, 255)} & \texttt{numpy}'s saturation/clamping function\\
\mintinline{python}{i = i.astype(np.float32)} & Converts the image type to \mintinline{python}{float32} (vs.\ \mintinline{python}{uint8, float64})\\
\mintinline{python}{x, _, _, _ = linalg.lstsq(A, b)} & Solves the least squares problem $\frac{1}{2}\lVert Ax - b\rVert^2$\\
\mintinline{python}{i = hstack((i1, i2))} & Merges $I_1$ and $I_2$ side-by-side\\
\mintinline{python}{i = vstack((i1, i2))} & Merges $I_1$ above $I_2$ \\
\mintinline{python}{i = fliplr(i)} & Flips image left-right\\
\mintinline{python}{i = flipud(i)} & Flips image up-down\\
\mintinline{python}{i = pad(i, ((1, 1), (3, 3)), 'reflect')} & Alternative to \mintinline{python}{copyMakeBorder} (also top, bottom, left, right)\\
\mintinline{python}{idx = argmax(i)} & Linear index of maximum in $I$ (i.e.\ index of flattened $I$)\\
\mintinline{python}{r, c = unravel_index(idx, i.shape)} & 2D coordinate of the index with respect to shape of \texttt{i}\\
\texttt{b = any(M > 5)} & Returns \mintinline{python}{True} if any element in array $M$ is greater than 5\\
\texttt{b = all(M > 5)} & Returns \mintinline{python}{True} if all elements in array $M$ are greater than 5\\
\texttt{rows, cols = where(M > 5)} & Returns indices of the rows and cols where elems in $M$ are >5\\
\mintinline{python}{coords = list(zip(rows, cols))} & Creates a list with the elements of \texttt{rows} and \texttt{cols} paired\\
\mintinline{python}{M_inv = linalg.inv(M)} & Inverse of $M$\\
\mintinline{python}{rad = deg2rad(deg)} & Converts degrees into radians\\
\end{tabular}
\subsubsection{Matplotlib.pyplot (\mintinline{python}{plt.})}
\begin{tabular}{@{}ll@{}}
\mintinline{python}{imshow(i, cmap="gray", vmin=0, vmax=255)} & \mintinline{python}{matplotlib}'s \mintinline{python}{imshow} preventing auto-normalization\\
\mintinline{python}{quiver(xx, yy, i_x, -i_y, color="green")} & Plots the gradient direction at positions \mintinline{python}{xx, yy}\\
\mintinline{python}{savefig("name.png")} & Saves the plot as an image\\
\end{tabular}





\rule{0.3\linewidth}{0.25pt}
\scriptsize

Copyright \copyright\ 2019 António Anjos (Rev: \today)\\
Most up-to-date version: \url{https://github.com/a-anjos/python-opencv}



\end{multicols}
\end{document}
